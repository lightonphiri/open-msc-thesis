\section{Research questions}
\label{sec:introduction:research-questions}

%%%%%Ma research questions yonse apa\ldots

The core of this research was aimed at investigating the feasibility of implementing a \gls{dls} based on simplified architectural designs. In particular, the research was guided by the following research questions:

\subsubsection*{Is it feasible to implement a \gls{dls} based on simple architectures?}
 
This primary research question was broadly aimed at investigating the viability of simple architectures. To this end, the following secondary questions were formulated to clarify the research problem.

\begin{enumerate}[label=\roman*]
\item \subsubsection*{How should simplicity for \gls{dls} storage and service architectures be defined?}
This research question served as a starting point for the research, and was devised to help provide scope and boundaries of simplicity for \gls{dls} design.
\item \subsubsection*{What are the potential implications of simplifying \gls{dls}---adverse or otherwise?}
It was envisaged, from the onset, that simplifying the overall design of a \gls{dls} would potentially result in both desirable and undesirable outcomes. This research question was thus aimed at identifying the implications of simplifying \gls{dls} design.
\item \subsubsection*{What are some of the comparative advantages and disadvantages of simpler architectures to complex ones?}
A number of \gls{dls} architectures have been proposed over the past two decades, ranging from those specifically designed to handle complex objects to those with an overall goal of creating and distributing collection archives (see Section~\ref{sec:background:digital-libraries-software}). This research question was aimed at identifying some of the advantages and disadvantages of simpler architectures compared to well-established \gls{dl} architectures. This includes establishing how well simple architectures support the scalability collections.
\end{enumerate} 