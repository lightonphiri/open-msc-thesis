\chapter{Introduction\label{ch:intro}}

%%%%%The introduction starts here\ldots

\begin{figure}
 \centering
 \framebox[\textwidth]{
 % Generated with LaTeXDraw 2.0.8
% Thu Mar 07 20:25:29 SAST 2013
% \usepackage[usenames,dvipsnames]{pstricks}
% \usepackage{epsfig}
% \usepackage{pst-grad} % For gradients
% \usepackage{pst-plot} % For axes
\scalebox{1} % Change this value to rescale the drawing.
{
\begin{pspicture}(0,-5.4)(12.38,5.24)
\definecolor{color2894b}{rgb}{0.9137254901960784,1.0,1.0}
\definecolor{color2639b}{rgb}{0.7176470588235294,0.8862745098039215,0.9411764705882353}
\psframe[linewidth=0.04,linecolor=color2894b,dimen=outer,shadow=true,shadowangle=-45.0,shadowsize=0.1,fillstyle=solid,fillcolor=color2894b](12.0,5.0)(0.0,-5.0)
\psframe[linewidth=0.06,linecolor=color2639b,dimen=outer,shadow=true,shadowangle=135.0,fillstyle=solid,fillcolor=color2639b](11.2,2.0)(0.6,-1.6)
\psframe[linewidth=0.06,linecolor=color2639b,dimen=outer,shadow=true,shadowangle=135.0,fillstyle=solid,fillcolor=color2639b](4.0,1.6)(1.2,0.8)
\psframe[linewidth=0.06,linecolor=color2639b,dimen=outer,shadow=true,shadowangle=135.0,fillstyle=solid,fillcolor=color2639b](4.0,0.6)(1.2,-0.2)
\psframe[linewidth=0.06,linecolor=color2639b,dimen=outer,shadow=true,shadowangle=135.0,fillstyle=solid,fillcolor=color2639b](4.0,-0.4)(1.2,-1.2)
\psframe[linewidth=0.06,linecolor=color2639b,dimen=outer,shadow=true,shadowangle=135.0,fillstyle=solid,fillcolor=color2639b](11.2,-2.55)(0.6,-4.55)
\usefont{T1}{ptm}{b}{n}
\rput(2.6096876,1.205){OAI-PMH}
\usefont{T1}{ptm}{b}{n}
\rput(2.6279688,0.205){OpenSearch}
\usefont{T1}{ptm}{b}{n}
\rput(2.553125,-0.795){SWORD}
\psframe[linewidth=0.06,linecolor=color2639b,dimen=outer,shadow=true,shadowangle=135.0,fillstyle=solid,fillcolor=color2639b](8.7,-0.4)(4.5,-1.2)
\usefont{T1}{ptm}{b}{n}
\rput(6.5421877,-0.795){Value-added Services}
\psframe[linewidth=0.06,linecolor=color2639b,dimen=outer,shadow=true,shadowangle=135.0,fillstyle=solid,fillcolor=color2639b](6.5,1.6)(4.5,0.8)
\usefont{T1}{ptm}{b}{n}
\rput(5.451875,1.205){Browse}
\psframe[linewidth=0.06,linecolor=color2639b,dimen=outer,shadow=true,shadowangle=135.0,fillstyle=solid,fillcolor=color2639b](6.5,0.8)(4.5,0.0)
\usefont{T1}{ptm}{b}{n}
\rput(5.51125,0.405){Ingestion}
\psframe[linewidth=0.06,linecolor=color2639b,dimen=outer,shadow=true,shadowangle=135.0,fillstyle=solid,fillcolor=color2639b](8.7,1.6)(6.7,0.8)
\usefont{T1}{ptm}{b}{n}
\rput(7.705625,1.205){Search}
\psframe[linewidth=0.06,linecolor=color2639b,dimen=outer,shadow=true,shadowangle=135.0,fillstyle=solid,fillcolor=color2639b](8.7,0.8)(6.7,0.0)
\usefont{T1}{ptm}{b}{n}
\rput(7.713125,0.405){Indexing}
\psframe[linewidth=0.06,linecolor=color2639b,dimen=outer,shadow=true,shadowangle=135.0,fillstyle=solid,fillcolor=color2639b](5.6,-3.1)(1.2,-4.1)
\usefont{T1}{ptm}{b}{n}
\rput(3.0798438,-3.595){Bitstream Objects}
\psframe[linewidth=0.06,linecolor=color2639b,dimen=outer,shadow=true,shadowangle=135.0,fillstyle=solid,fillcolor=color2639b](10.4,-3.1)(6.0,-4.1)
\usefont{T1}{ptm}{b}{n}
\rput(7.97,-3.595){Metadata Objects}
\psframe[linewidth=0.06,linecolor=color2639b,dimen=outer,shadow=true,shadowangle=135.0,fillstyle=solid,fillcolor=color2639b](11.2,4.6)(0.6,3.0)
\psframe[linewidth=0.06,linecolor=color2639b,dimen=outer,shadow=true,shadowangle=135.0,fillstyle=solid,fillcolor=color2639b](5.6,4.3)(1.2,3.3)
\usefont{T1}{ptm}{b}{n}
\rput(3.37125,3.805){Machine Interaction}
\psframe[linewidth=0.06,linecolor=color2639b,dimen=outer,shadow=true,shadowangle=135.0,fillstyle=solid,fillcolor=color2639b](10.6,4.3)(6.2,3.3)
\usefont{T1}{ptm}{b}{n}
\rput(8.350625,3.805){User Interaction}
\psline[linewidth=0.06cm,linecolor=darkgray,linestyle=dashed,dash=0.16cm 0.16cm](4.2,1.6)(4.2,-1.2)
\psline[linewidth=0.06cm,linecolor=darkgray,linestyle=dashed,dash=0.16cm 0.16cm](4.6,-0.1)(8.6,-0.1)
\psframe[linewidth=0.06,linecolor=color2639b,dimen=outer,shadow=true,shadowangle=-135.0,fillstyle=solid,fillcolor=color2639b](10.7,1.6)(9.3,-1.2)
\usefont{T1}{ptm}{b}{n}
\rput{90.0}(10.008125,-9.642187){\rput(9.811093,0.205){Object}}
\usefont{T1}{ptm}{b}{n}
\rput{90.0}(10.492031,-10.131407){\rput(10.304531,0.205){Management}}
\psline[linewidth=0.06cm,linecolor=darkgray,linestyle=dashed,dash=0.16cm 0.16cm](9.0,1.6)(9.0,-1.2)
\end{pspicture} 
}


 }
 \caption{High level architecture of a typical Digital Library System}
 \label{fig:introduction:overview:digital-library-system-architecture}
\end{figure}

The last few decades has seen an overwhelming increase in the amount of digitised and born digital information. There has also been a growing need for specialised systems tailored to better handle this digital content. \glspl{dl} are specifically designed to store, manage and preserve digital objects over long periods of time. Figure~\ref{fig:introduction:overview:digital-library-system-architecture} illustrates a high-level view of a typical \gls{dls} architecture.

% include other files for sections of this chapter. These use the 'input' command since each section within a chapter should not start a new page.
% If you want to swap the order of sections, it is as simple as reversing the order you include them. 
\section{Motivation}
\label{sec:introduction:motivation}

%%%%%Motivation ifunika apa\ldots

\glspl{dl} began as an abstraction layered over databases to provide higher level services \citep{Arms1997,Baldonado1997,Frew1998} and have evolved, subsequently making them complex \citep{Janee2002,Lagoze2006} and difficult to maintain, extend and reuse. The difficulties resulting from the complexities of such tools are especially prominent in organisations and institutions that have limited resources to manage such tools and services. Some examples of organisations that fall within this category include cultural heritage organisations and a significant number of other organisations in developing countries found in regions such as Africa \citep{Suleman2008}.

The majority of existing platforms are arguably unsuitable for resource-constrained environments due to the following reasons:

\begin{itemize}
 \item Some organisations do not have sustainable funding models, making it difficult to effectively manage the preservation life-cycle as most tools are composed of custom and third-party components that require regular updates.
 \item A number of existing tools require technically-inclined experts to manage them, effectively raising their management costs.
 \item The majority of modern platforms are bandwidth intensive. However, they sometimes end up being deployed in regions were Internet bandwidth is unreliable and mostly very expensive, making it difficult to guarantee widespread accessibility to services offered.
\end{itemize}

A potential solution to this problem is to explicitly simplify the overall design of \glspl{dls} so that the resulting tools and services are more easily adopted and managed over time. This premise is drawn from the many successes of the application of minimalism, as discussed in Section~\ref{sec:background:simple-architectures}. In light of that, this research proposes the design of lightweight tools and services, with the potential to be easily adopted and managed.
%%%%%\section{Problem statement}
\label{sec:intro:problem-statement}

Vuti ilikuti\ldots
\section{Hypotheses}
\label{sec:ch-intro:hypotheses}

%%%%%Hypotheses kuyachita list apa\ldots

This research was guided by three working hypotheses that are a direct result of grounding work previously conducted \citep{Suleman2007,Suleman2010a}. The three hypotheses are as follows:

\begin{itemize}
 \item A formal simplistic abstract framework for \gls{dls} design can be derived.
 \item A \gls{dls} architectural design based on a simple and minimalistic approach could be potentially easy to adopt and manage over time.
 \item The system performance of tools and services based on simple architectures could be adversely affected.
\end{itemize}
\section{Research questions}
\label{sec:introduction:research-questions}

%%%%%Ma research questions yonse apa\ldots

The core of this research was aimed at investigating the feasibility of implementing a \gls{dls} based on simplified architectural designs. In particular, the research was guided by the following research questions:

\subsubsection*{Is it feasible to implement a \gls{dls} based on simple architectures?}
 
This primary research question was broadly aimed at investigating the viability of simple architectures. To this end, the following secondary questions were formulated to clarify the research problem.

\begin{enumerate}[label=\roman*]
\item \subsubsection*{How should simplicity for \gls{dls} storage and service architectures be defined?}
This research question served as a starting point for the research, and was devised to help provide scope and boundaries of simplicity for \gls{dls} design.
\item \subsubsection*{What are the potential implications of simplifying \gls{dls}---adverse or otherwise?}
It was envisaged, from the onset, that simplifying the overall design of a \gls{dls} would potentially result in both desirable and undesirable outcomes. This research question was thus aimed at identifying the implications of simplifying \gls{dls} design.
\item \subsubsection*{What are some of the comparative advantages and disadvantages of simpler architectures to complex ones?}
A number of \gls{dls} architectures have been proposed over the past two decades, ranging from those specifically designed to handle complex objects to those with an overall goal of creating and distributing collection archives (see Section~\ref{sec:background:digital-libraries-software}). This research question was aimed at identifying some of the advantages and disadvantages of simpler architectures compared to well-established \gls{dl} architectures. This includes establishing how well simple architectures support the scalability collections.
\end{enumerate} 
\section[Scope\& approach]{Scope and approach}
\label{sec:ch-intro:approach}

%%%%%Approach nayeve pamene apa\ldots

Table~\ref{tab:introduction:scope-and-approach:research-approach} shows a summary of the research process followed to answer the research questions.

\tablespacing
%%%%%\begin{longtable}{p{0.30\linewidth} p{0.60\linewidth}}
\begin{longtable}{
>{\arraybackslash}p{0.30\linewidth}|
>{\arraybackslash}p{0.60\linewidth}}
 
 \caption{Summary of research approach process}
\label{tab:introduction:scope-and-approach:research-approach} \\
 %%%%%\toprule
 %%%%%\hline
 \textbf{Research Process} & \textbf{Procedure}\\
 %%%%%\midrule
 \cline{1-2}
 \endfirsthead
 
 \caption[]{(continued)}\\
 %%%%%\toprule
 %%%%%\hline
 \textbf{Research Process} & \textbf{Procedure}\\
 %%%%%\midrule
 \cline{1-2}
 \endhead
 
 % Page footer
 %%%%%\midrule
 %%%%%\hline
 \multicolumn{2}{r}{(Continued on next page)} \\
 \endfoot
 
 % Last page footer
 %%%%%\bottomrule
 \endlastfoot
 
 \textbf{Literature synthesis} &
 {Preliminary review of existing literature} \\
 
 \cline{1-2}
 %\cmidrule[0.1pt](l{0.5em}r{0.5em}){1-2}
 
 \textbf{Research proposal} &
 {Scoping and formulation of research problem} \\
 
 \cline{1-2}
 %\cmidrule[0.1pt](l{0.5em}r{0.5em}){1-2}
 
 \textbf{Exploratory study} &
 {Derivation of design principles} \\
 
 \cline{1-2}
 %\cmidrule[0.1pt](l{0.5em}r{0.5em}){1-2}
 
 \textbf{Repository design} &
 {Mapping of design principles to design process} \\
 
 \cline{1-2}
 %\cmidrule[0.1pt](l{0.5em}r{0.5em}){1-2}

 \textbf{Case studies} &
 {Implementation case study collections} \\
 
 \cline{1-2}
 %\cmidrule[0.1pt](l{0.5em}r{0.5em}){1-2}

 \textbf{Evaluation} &
 {Experimentation results and discussion} \\
 
\end{longtable}

\bodyspacing
\section{Thesis outline}
\label{sec:ch-intro:thesis-outline}

%%%%%Outline iyambila apa\ldots

This manuscript is structured as follows:

\begin{itemize}
 \item Chapter~\ref{ch:intro} serves as an introduction, outlining the motivation, research questions and scope of the research conducted.
 \item Chapter~\ref{ch:background} provides background information and related work relevant to the research conducted.
 \item In Chapter~\ref{ch:exploratory-study} the exploratory study that was systematically conducted to derive a set of design principles is described, including the details of the principles derived.
 \item Chapter~\ref{ch:implementation} presents a prototype repository whose design decisions are directly mapped to some design principles outlined in Chapter~\ref{ch:exploratory-study}.
 \item Chapter~\ref{ch:case-studies} describes two real-world case study implementation designed and implemented using the repository design outlined in Chapter~\ref{ch:implementation}.
 \item The implications of the prototype repository design are outlined in Chapter~\ref{ch:evaluation} through: experimental results from a developer-oriented survey conducted to evaluate the simplicity and extensibility; and through scalability performance benchmark results of some \gls{dls} operations conducted on datasets of different sizes.
 \item Chapter~\ref{ch:conclusion} highlights concluding remarks and recommendations for potential future work.
\end{itemize}

%%%%%http://worldstarhiphop.com/videos/video.php?v=wshh58i7F3dJBw75knrB
