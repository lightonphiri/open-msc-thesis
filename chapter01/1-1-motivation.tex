\section{Motivation}
\label{sec:introduction:motivation}

%%%%%Motivation ifunika apa\ldots

\glspl{dl} began as an abstraction layered over databases to provide higher level services \citep{Arms1997,Baldonado1997,Frew1998} and have evolved, subsequently making them complex \citep{Janee2002,Lagoze2006} and difficult to maintain, extend and reuse. The difficulties resulting from the complexities of such tools are especially prominent in organisations and institutions that have limited resources to manage such tools and services. Some examples of organisations that fall within this category include cultural heritage organisations and a significant number of other organisations in developing countries found in regions such as Africa \citep{Suleman2008}.

The majority of existing platforms are arguably unsuitable for resource-constrained environments due to the following reasons:

\begin{itemize}
 \item Some organisations do not have sustainable funding models, making it difficult to effectively manage the preservation life-cycle as most tools are composed of custom and third-party components that require regular updates.
 \item A number of existing tools require technically-inclined experts to manage them, effectively raising their management costs.
 \item The majority of modern platforms are bandwidth intensive. However, they sometimes end up being deployed in regions were Internet bandwidth is unreliable and mostly very expensive, making it difficult to guarantee widespread accessibility to services offered.
\end{itemize}

A potential solution to this problem is to explicitly simplify the overall design of \glspl{dls} so that the resulting tools and services are more easily adopted and managed over time. This premise is drawn from the many successes of the application of minimalism, as discussed in Section~\ref{sec:background:simple-architectures}. In light of that, this research proposes the design of lightweight tools and services, with the potential to be easily adopted and managed.