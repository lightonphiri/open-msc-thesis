\section{Design decisions}\index{Software design decisions}
\label{sec:background:software-design-decisions}

Software architectures provide an overview of a software system's components, and the relationships and characteristics that exist between the various components \citep{Lee2007}. The architectures are initially conceived as a composition of the general design, influenced by a corresponding set of design decisions \citep{Kruchten2005}. The design decisions form a fundamental part of the architectural design process, by guiding the development of the software product, as they help ensure that the resulting product conforms to desired functional and non-functional requirements.

There are two prominent methods---design rationale and formalised ontological representation---used to capture design decisions \citep{Lee2007}. The design rationale\index{Design Rationale} provides a historical record, in form of documentation, of the rationale used to arrived at a particular design approach \citep{Lee1991}, and typically makes use of techniques such as Issue-Based Information Systems (IBIS)\index{Software Design Decisions!Design Rationale!IBIS} \citep{Conklin1991} and ``Questions, Options and Criteria'' (QOC)\index{Software Design Decisions!Design Rationale!QOC} \citep{MacLean1991}. The formalised ontological representation method\index{Software Design Decisions!Formalised Ontological Representation} on the other hand makes use of an ontological model for describing and categorising the architectural design decisions \citep{Kruchten2004}.

There are a number of benefits of explicitly capturing and documenting design decisions, the most significant one being that they help in---ensuring the development of the desired product. In the case of domain-specific products, they form a crucial role of ensuring that the resulting solution directly conforms to the solution space it is meant to operate within.

%http://www.w3.org/DesignIssues/Principles.html
%ftp://ftp.isi.edu/in-notes/rfc1958.txt


%-> 2 main methods for documenting design decisions [5]:
%  (a) Design Rationale
%  
%  (b) Ontological

%[1] http://www.cs.ubc.ca/~gregor/teaching/papers/4+1view-architecture.pdf
%[5] http://ieeexplore.ieee.org/stamp/stamp.jsp?tp=&arnumber=4232835
%[6] P. Kruchten, P. Lago, H. van Vliet and T. Wolf, “Building up and exploiting
%architectural knowledge,” 5th IEEE/IFIP Working Conference on Software
%Architecture, 2005.
