\section{Data storage schemes}
\label{sec:background:data-storage-architectures}

The repository sub-layer forms the core architectural component of a typical
digital library system and more specifically, it is composed of two components:
a bitstream\index{Bitstream} store and a metadata\index{Metadata} store, responsible for storing digital content
and metadata\index{Metadata} records respectively. As shown in Table~\ref{tab:background:related-work:digital-libraries-software:floss-matrix},
\glspl{dls} are generally implemented in such a manner that digital
content is stored on the file system, whilst the metadata\index{Metadata} records are almost
always housed in a relational database\index{RDMS}.

This section discusses three prominent data storage solutions that can
potentially be integrated within the repository sub-layer for metadata\index{Metadata} storage.
The focus is to assess their suitability for integration with
\glspl{dls}.

\subsection{Relational databases}
\label{sec:background:data-storage-architectures:relational-databases}

Relational databases\index{Storage Schemes!RDBMS} have stood the test of time, having been around for
decades. They have, until recently, been the preferred choice for data
storage. There are a number of reasons \citep[see][chap. 3]{Elmasri2008} why relational databases have proved to
be a popular storage solution, and these include:

\begin{itemize}
 \item The availability of a simple, but effective query language---SQL---
capable of retrieving multifaceted views of data
 \item Support for Data model relationships via table relations
 \item Transaction support through ACID\footnote{Atomicity, Consistency, Isolation, Durability} properties
 \item Support for data normalisation, thus preventing redundancy
\end{itemize}

Relational databases are, however, mostly suitable for problem domains
that require frequent retrieval and update of relatively small quantities of
data.

\subsection{NoSQL databases}
\label{sec:background:data-storage-architectures:nosql-databases}

The large-scale production of data \citep{Gantz2008}, coupled with the now
prevalent Big Data\footnote{\url{http://www-01.ibm.com/software/data/bigdata}},
has resulted in a profound need for data storage architectures that are
efficient, horizontally scalable, and easier to interface with. As a result,
NoSQL databases recently emerged as potential alternatives to relational
databases. NoSQL\index{Storage Schemes!NoSQL} databases are non-relational databases that embrace schemaless
data, are capable of running on clusters, and generally trade off consistency
for other properties such as performance \citep[see][chap.
1]{Sadalage2004}.

NoSQL database implementations are often categorised based on the manner in
which they store data, and typically fall under the categories described in
Table~\ref{tab:background:data-storage-architectures:nosql-database-data-models}.

\tablespacing
%%%%%\begin{longtable}{p{0.30\linewidth} p{0.60\linewidth}}
\begin{longtable}{
>{\arraybackslash}p{0.32\linewidth}|
>{\arraybackslash}p{0.58\linewidth}}
 
 \caption{Data model categories for NoSQL database stores}
\label{tab:background:data-storage-architectures:nosql-database-data-models} \\
 %%%%%\toprule
 %%%%%\hline
 \textbf{Data Model} & \textbf{Description}\\
 %%%%%\midrule
 \cline{1-2}
 \endfirsthead
 
 \caption[]{(continued)}\\
 %%%%%\toprule
 %%%%%\hline
 \textbf{Data Model} & \textbf{Description}\\
 %%%%%\midrule
 \cline{1-2}
 \endhead
 
 % Page footer
 %%%%%\midrule
 %%%%%\hline
 \multicolumn{2}{r}{(Continued on next page)} \\
 \endfoot
 
 % Last page footer
 %%%%%\bottomrule
 \endlastfoot
 
 \textbf{Column-Family Stores} &
 {Data is stored with keys mapped to values grouped into column families} \\
 
 \cline{1-2}
 %\cmidrule[0.1pt](l{0.5em}r{0.5em}){1-2}
 
 \textbf{Document Stores} &
 {Data is stored in self-describing encoded data structures} \\
 
 \cline{1-2}
 %\cmidrule[0.1pt](l{0.5em}r{0.5em}){1-2}
 
 \textbf{Graph Stores} &
 {Data is stored as entities with corresponding relationships between entities} \\
 
 \cline{1-2}
 %\cmidrule[0.1pt](l{0.5em}r{0.5em}){1-2}
 
 \textbf{Key-Value Stores} &
 {Hash table with unique keys and corresponding pointer to blobs} \\
 
\end{longtable}

\bodyspacing

NoSQL databases are highly optimised for retrieve and append operations and,
as a result, there has recently been an increase in the number of applications
that are making use of NoSQL data stores. However, the downside of NoSQL
databases is that they cannot simultaneously guarantee data consistency, availability and partition tolerance; as defined in the CAP theorem
\citep{Gilbert2002}.

\subsection{Filesystems}
\label{sec:background:data-storage-architectures:file-systems}

File systems\index{Storage Schemes!File Systems} are implemented by default in all operating systems, and provide a
persistent store for data. In addition, they provide a means to organise data
in a manner that facilitates subsequent retrieval and update of data.

Native file systems have, in the past, not generally been used as storage layers
for enterprise applications, in part, due to the fact that they do not provide
explicit support for transaction management and fast indexing of data. However,
the emergence of clustered environments has resulted in robust and
reliable distributed file system technologies such as Apache Hadoop
\citep{Borthakur2007} and Google File System \citep{Ghemawat2003}.

The opportunities presented by traditional file systems, and in particular their
simplicity, efficiency and general ease of customisation make them prime
candidates for storage of both digital content and metadata\index{Metadata} records. In
addition, the use of flat files, and more specifically text files, for storage
of metadata\index{Metadata} records could further complement and simplify the digital
library repository sub-layer. Incidentally, Raymond \citep[see][chap.
5]{Raymond2004,} highlights a number of advantages associated with using text
files, and further emphasises that designing textual protocols ultimately
results in future-proof systems.

In general, there are a number of real-word application whose data storage
implementations take advantage of file systems. Some notable example
implementations of both digital libraries specific tools and general purpose
tools are outlined below.

\paragraph{BagIt file packaging format}

The BagIt\index{File-based Stores!BagIt} File Packaging Format specification \citep{Boyko2012} defines a
hierarchical file packaging format suitable for exchanging digital content. The
BagIt format is streamlined for disk-based and network-based storage and
transfer. The organisation of bags is centred on making use of file system
directories as bags, which at a minimum contain: a data directory, at least one
manifest file that lists data directory contents, and a bagit.txt file that
identifies the directory as a bag.

\paragraph{DokuWiki}

DokuWiki\index{File-based Stores!DokuWiki} is a PHP based Wiki engine, mainly aimed at creating documentation, that is standards compliant and easy to use \citep{DokuWiki}. The storage architecture of DokuWiki principally makes use of the filesystem as its data store, with application data files stored in plain text files. This design strategy ensures that data is accessible even when the server goes down, and at the same time facilities backup and restore operations through the use of basic server scripts and FTP/sFTP.

\paragraph{Git}

% Pretty neat git object structure information
% http://git-scm.com/book/en/Git-Internals-Git-Objects
Git is a distributed version control system that functions as
a general tool for filesystem directory content tracking, and is designed
with a strong focus on speed, efficiency and real-world usability on
large projects \citep[see][chap. 1]{Chacon2009}, to attain three core functional requirements below.

\begin{itemize}
 \item Store generic content
 \item Track content changes in the repository
 \item Facilitate a distributed architecture for the content
\end{itemize}

Git\index{File-based Stores!Git} is internally represented as a duplex data structure that is composed of a
mutable index for caching information about the working directory; and an
immutable repository. The Git object storage area is a Directed Acyclic Graph
that is composed of four types of objects---blob objects; tree objects;
commit objects; and tag objects \citep{VirtanenGit4CS}. In addition, the repository is
implemented as a generic content-addressable filesystem with objects stored in a
simple key-value data store \citep[see][chap.9]{Chacon2009}.

\subparagraph{Pairtrees for collection storage}

Pairtree\index{File-based Stores!Pairtree} is a file system convention for organising digital object stores, and
has the advantage of making it possible for object specific operations to be
performed by making use of native operating system tools \citep{Pairtrees}.

\subsection{Summary}
\label{sec:background:file-based-storage:summary}

There are numerous available data storage options, and it is
important to understand the varying options to fully identify the ones most
applicable to specific problem domains. It is generally not always the case
that a definitive storage solution is arrived at, however, a data model that
better matches the kind of data storage and retrieval requirements should be the
primary deciding factor. Table~\ref{tab:background:data-storage-architectures:comparison} is a
summarised comparative matrix outlining the three storage solutions
discussed in this section.

It is that it is generally not always necessary to use an intermediate data management infrastructure, and in some cases, it may in all actuality be desirable not to use one at all; as is the case with the real world applications described in Section~\ref{sec:background:data-storage-architectures:file-systems}.

\tablespacing
%%%%%\begin{longtable}{p{0.20\linewidth} p{0.20\linewidth} p{0.20\linewidth}
%%%%%p{0.20\linewidth}}
\begin{longtable}{
>{\arraybackslash}p{0.20\linewidth}
|>{\arraybackslash}p{0.20\linewidth}
|>{\arraybackslash}p{0.20\linewidth}
|>{\arraybackslash}p{0.20\linewidth}}
 
 \caption{Comparative matrix for data storage solutions}
\label{tab:background:data-storage-architectures:comparison} \\
 %%%%%\toprule
 %%%%%\cline{2-4}
 \multicolumn{1}{c|}{} & 
 \textbf{File Systems} & 
 \textbf{RDBMS} &
 \textbf{NoSQL} \\
 %%%%%\midrule
 \cline{1-4}
 \endfirsthead
 
 \caption[]{(continued)}\\
 %%%%%\toprule
 %%%%%\cline{2-4}
 \multicolumn{1}{c|}{} & 
 \textbf{File Systems} & 
 \textbf{RDBMS} &
 \textbf{NoSQL} \\
 %%%%%\midrule
 \cline{1-4}
 \endhead
 
 % Page footer
 %%%%%\midrule
 \hline
 \multicolumn{4}{r}{(Continued on next page)} \\
 \endfoot
 
 % Last page footer
 %%%%%\bottomrule
 \endlastfoot
 
 \textbf{Use Cases} &
 {Miscellaneous} & 
 {Relational data} &
 {Large-scale data} \\
 
 \cline{1-4}
 %%%%%\cmidrule[0.1pt](l{0.5em}r{0.5em}){1-4}
 
 \textbf{Data Format} &
 {Heterogeneous data} & 
 {Structured data} &
 {Unstructured data} \\
 
 \cline{1-4}
 %%%%%\cmidrule[0.1pt](l{0.5em}r{0.5em}){1-4}
 
 \textbf{Transaction Support} &
 {Simple Locking} & 
 {ACID compliant} & 
 {CAP theorem support} \\
 
 \cline{1-4}
 %%%%%\cmidrule[0.1pt](l{0.5em}r{0.5em}){1-4}
 
 \textbf{Indexing} &
 {Optional} & 
 {Available} & 
 {Available} \\
 
 \cline{1-4}
 %%%%%\cmidrule[0.1pt](l{0.5em}r{0.5em}){1-4}
 
 \textbf{Scalability} &
 {Horizontal; Vertical} & 
 {Vertical} & 
 {Horizontal} \\
 
 \cline{1-4}
 %%%%%\cmidrule[0.1pt](l{0.5em}r{0.5em}){1-4}
 
 \textbf{Replication}&
 {Partial support} & 
 {Explicit support} & 
 {Explicit support} \\
 
\end{longtable}

\bodyspacing