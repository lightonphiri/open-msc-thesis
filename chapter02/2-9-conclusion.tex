\section{Summary}
\label{sec:background:conclusion}

This chapter discussed background information that forms the basis for this research. Section~\ref{sec:background:digital-libraries} discussed \glspl{dl}, through elaborate high level definitions, complemented with examples of varying application domains within which contemporary \glspl{dls} are utilised. Core fundamental concepts associated to \gls{dl} were also discussed in ~\ref{sec:background:fundamental-concepts}.

Some prominent \gls{dl} frameworks were presented in Section~\ref{sec:background:reference-models-frameworks}, followed by \gls{floss} \gls{dl} software tools in Section~\ref{sec:background:digital-libraries-software}; revealing that the varying frameworks and architectural designs are largely as a result of the different problems for which solutions were sought. However, there are core features that are common to most of the proposed solutions. It could thus be argued that existing solutions may not be be suitable for certain environments, and as such simpler alternative architectural designs may be desirable. A culmination of the argument for utilising simpler architectural designs manifested in the discussion of prominent designs that used simplicity as the core criterion in Section~\ref{sec:background:simple-architectures}. 

In addition, the repository sub-layer was highlighted as the component that forms the core of digital libraries in Section~\ref{sec:background:data-storage-architectures}, and a further discussing of potential storage solutions that can be used for the storage of metadata records then followed. Traditional file systems have been identified as contenders of the more generally accepted relational databases and now common place NoSQL databases, for the storage of metadata records.

Furthermore, a discussion of two major general approaches followed when arriving at software design decisions were presented in ~\ref{sec:background:software-design-decisions}.

%In conclusion, the design of \glspl{dls} requires a well structured formal process to ensure an all-inclusive final system.