%%\section{Overview}
%%\label{sec:evaluation-overview}
Evaluation of \glspl{dl}\index{Digital Libraries} has been a subject of interest for \gls{dl}\index{Digital Libraries} research from the very early stages. This is evidenced by early initiatives such as the D-Lib Working Group on \gls{dl}\index{Digital Libraries} Metrics \citep{DLWG1998} that was established in the late 1990s. A series of related studies have since been conducted with the aim of outlining a systematic and viable way of evaluating the complex, multi-faceted nature of \glspl{dl}\index{Digital Libraries} that encompasses content, system and user-oriented aspects. For instance, the DELOS\index{DELOS}\footnote{\url{http://www.delos.info}} Cluster on Evaluation \citep{Borgman2002,DELOSW72004}, which is perhaps the most current and comprehensive \gls{dl}\index{Digital Libraries} evaluation initiative, was initiated with the aim of addressing the different aspects of \glspl{dl}\index{Digital Libraries} evaluation.

The DELOS\index{DELOS} \gls{dl}\index{Digital Libraries} evaluation activities have yielded some significant results; in an attempt to understand the broad view of \glspl{dl}, Fuhr et al. \citep{Fuhr2001} developed a classification and evaluation scheme using four major dimensions: data/collection, system/technology, users and usage, and further produced a MetaLibrary comprising of test-beds to be used in \gls{dl}\index{Digital Libraries} evaluation. In a follow-up paper, Fuhr et al. \citep{Fuhr2007} proposed a new framework for evaluation of \glspl{dl}\index{Digital Libraries} with detailed guidelines for the evaluation process.

This research proposes simplifying the overall design of \glspl{dls}\index{Digital Library System} and more specifically designing for simplicity of management and ease of use the resulting \glspl{dls}\index{Digital Library System}. The design principles derived in Chapter~\ref{ch:exploratory-study} were used to design and implement a simple generic repository sub-layer for \glspl{dls}\index{Digital Library System}. In Chapter~\ref{ch:case-studies} three proof of concept file-based repository implementations are presented to evaluate the effectiveness of this approach. 

A developer survey, outlined in Section~\ref{sec:evaluation:developer-survey}, was conducted to assess the impact of redesigning the repository sub-layer on extensibility of implementations based on this design. 

Furthermore, owing to the fact that repositories have a potential to grow, detailed scalability performance benchmarks were conducted to assess the performance of this design strategy relative to the sizes of collections; these performance benchmarks are outlined in Section~\ref{sec:evaluation:performance}.