\section{Research methods}
\label{sec:exploratory-study:research-methods}

\subsection{Grounded theory}\index{Grounded theory}
\label{sec:exploratory-study:grounded-theory}

Grounded Theory is a research method that provides a technique for developing
theory iteratively from qualitative data. The goal of Grounded Theory is to
generate a theory that accounts for a pattern of behaviour that is relevant for those involved \citep{Glaser1978}. Grounded Theory attempts to find the main concern of the participants and how they go about resolving it, through constant comparison of data at increasing levels of abstraction and has also been described as ``a general pattern for understanding'' \citep{Glaser1992}.

The Grounded Theory method generally revolves around a series of five
steps, as outlined below.

\subsubsection{Grounded theory process}

\paragraph{Step 1} Data collection

This step uses a method appropriate to the research context to elicit
information from selected participants. Typical methods include
conducting semi-structured interviews.

\paragraph{Step 2} Data analysis

The data analysis step forms the core of grounded theory and generally involves
the use of a constant comparative method to generate and analyse data.

\paragraph{Step 3} Memoing\index{Memoing}

Memoing, as the name suggests, involves writing theoretical memos to identify
relationships between different patterns of the data.

\paragraph{Step 4} Sorting

The sorting step takes the form of arranging all memos once the data collection
becomes saturated. The outcome of this results in a theory describing how the
identified categories relate to the core category.

\paragraph{Step 5} Theoretical coding

The data collected is divided into segments to identify categories or themes.
The categorised data is then further examined to identify properties common to
each of the categories.

Grounded theory was selected as the primary research method for the exploratory
study due to the following reasons:

\begin{itemize}
 \item It is primarily aimed at theory generation, focusing specifically on
generating theoretical ideas, explanations and understanding of the data.
 \item It is useful when trying to gain a fresh perspective of a
well-known area.
  \item It has proven to be a successful method for exploring human and social
aspects.
  \item It is by far one of the most common/popular analytic technique in
qualitative analysis.
  \item It is arguably intuitive.
\end{itemize}


\subsection{Analytic hierarchy process}\index{Analytic hierarchy process}
\label{sec:exploratory-study:ahp}

The \gls{ahp} is a theory of measurement through
pairwise comparisons that relies on judgement of experts to derive priority scales
\citep{Saaty2008}. A pairwise comparison is a problem-solving technique that
allows one to determine the the most significant item among a group of items. The overall
process is driven by scales of absolute judgement that represent how much more
an element dominates another with respect to a given attribute. The pairwise
comparison method involves following a series of steps and is outlined in
Section~\ref{sec:exploratory-study:research-methods:ahp:pairwise}.

\subsubsection{Pairwise comparisons method}\index{Pairwise comparisons}
\label{sec:exploratory-study:research-methods:ahp:pairwise}

The method of pairwise comparisons ensures that for a given set of elements or
alternatives, each candidate element is matched head to head with the other
candidates and is performed by decomposing decisions \citep{Saaty2008} into the
steps outlined below.

\paragraph{Step 1} Define the criteria to be ranked. 

The criteria identified are influenced by the overall objectives and form the
basis of the comparative analysis.

\paragraph{Step 2} Arrange the criteria in an $N\times N$ matrix.

In essence, each element in a given set a of $N$ elements is compared against
other alternatives in the set as shown in Table~\ref{tab:exploratory-study:ahp:matrix}. The total number of pairwise
comparisons can thus be computed using equation:

\begin{equation*}
%\label{eq:exploratory-study:research-methods:pairwise}
 \frac{N (N-1)}{2}
\end{equation*}

\tablespacing
%%%%%\begin{longtable}{p{0.05\linewidth} p{0.05\linewidth} p{0.05\linewidth}
%%%%%p{0.05\linewidth} p{0.05\linewidth} p{0.05\linewidth} p{0.05\linewidth}
%%%%%p{0.05\linewidth} p{0.05\linewidth} p{0.05\linewidth} p{0.05\linewidth}}
\begin{longtable}{
>{\centering\arraybackslash}m{0.05\linewidth}
|>{\centering\arraybackslash}m{0.05\linewidth}
|>{\centering\arraybackslash}m{0.05\linewidth}
|>{\centering\arraybackslash}m{0.05\linewidth}
|>{\centering\arraybackslash}m{0.05\linewidth}
|>{\centering\arraybackslash}m{0.05\linewidth}
|>{\centering\arraybackslash}m{0.05\linewidth}
|>{\centering\arraybackslash}m{0.05\linewidth}
|>{\centering\arraybackslash}m{0.05\linewidth}
|>{\centering\arraybackslash}m{0.05\linewidth}|
>{\centering\arraybackslash}m{0.05\linewidth}}

\caption{An $N\times N$ pairwise comparisons matrix}
\label{tab:exploratory-study:ahp:matrix} \\

 %%%%%\toprule
 \cline{3-10}
 \multicolumn{1}{c}{} &
 \multicolumn{1}{c|}{} &
 \textbf{H} &
 \textbf{G} &
 \textbf{F} &
 \textbf{E} &
 \textbf{D} &
 \textbf{C} &
 \textbf{B} &
 \textbf{A} &
 {} \\
 \cline{2-10}
 %%%%%\midrule
 \cline{2-10}
 \endfirsthead

 \caption[]{(continued)}\\
 %%%%%\toprule
 \cline{3-10}
 \multicolumn{1}{c}{} &
 \multicolumn{1}{c|}{} &
 \textbf{H} &
 \textbf{G} &
 \textbf{F} &
 \textbf{E} &
 \textbf{D} &
 \textbf{C} &
 \textbf{B} &
 \textbf{A} &
 {} \\
 %%%%%\midrule
 \cline{2-10}
 \endhead

 % Page footer
 %%%%%\midrule
 \cline{2-10}
 \multicolumn{9}{r}{(Continued on next page)} \\
 \endfoot

 % Last page footer
 %%%%%\bottomrule
 \endlastfoot

 \multicolumn{1}{c|}{}&
 \textbf{A}&
 {X}&
 {X}&
 {X}&
 {X}&
 {X}&
 {X}&
 {X}&
 \multicolumn{1}{c}{}&
 \multicolumn{1}{c}{}\\

 \cline{2-9}
 %\cmidrule[0.1pt](l{0.5em}r{0.5em}){1-7}

 \multicolumn{1}{c|}{}&
 \textbf{B}&
 {X}&
 {X}&
 {X}&
 {X}&
 {X}&
 {X}&
 \multicolumn{1}{c}{}&
 \multicolumn{1}{c}{}&
 \multicolumn{1}{c}{}\\

 \cline{2-8}
 %\cmidrule[0.1pt](l{0.5em}r{0.5em}){1-7}

 \multicolumn{1}{c|}{}&
 \textbf{C}&
 {X}&
 {X}&
 {X}&
 {X}&
 {X}&
 \multicolumn{1}{c}{}&
 \multicolumn{1}{c}{}&
 \multicolumn{1}{c}{}&
 \multicolumn{1}{c}{}\\

 \cline{2-7}
 %\cmidrule[0.1pt](l{0.5em}r{0.5em}){1-7}

 \multicolumn{1}{c|}{}&
 \textbf{D}&
 {X}&
 {X}&
 {X}&
 {X}&
 \multicolumn{1}{c}{}&
 \multicolumn{1}{c}{}&
 \multicolumn{1}{c}{}&
 \multicolumn{1}{c}{}&
 \multicolumn{1}{c}{}\\

 \cline{2-6}
 %\cmidrule[0.1pt](l{0.5em}r{0.5em}){1-7}

 \multicolumn{1}{c|}{}&
 \textbf{E}&
 {X}&
 {X}&
 {X}&
 \multicolumn{1}{c}{}&
 \multicolumn{1}{c}{}&
 \multicolumn{1}{c}{}&
 \multicolumn{1}{c}{}&
 \multicolumn{1}{c}{}&
 \multicolumn{1}{c}{}\\

 \cline{2-5}
 %\cmidrule[0.1pt](l{0.5em}r{0.5em}){1-7}
 
 \multicolumn{1}{c|}{}&
 \textbf{F}&
 {X}&
 {X}&
 \multicolumn{1}{c}{}&
 \multicolumn{1}{c}{}&
 \multicolumn{1}{c}{}&
 \multicolumn{1}{c}{}&
 \multicolumn{1}{c}{}&
 \multicolumn{1}{c}{}&
 \multicolumn{1}{c}{}\\
 
 \cline{2-4}

 \multicolumn{1}{c|}{}&
 \textbf{G}&
 {X}&
 \multicolumn{1}{c}{}&
 \multicolumn{1}{c}{}&
 \multicolumn{1}{c}{}&
 \multicolumn{1}{c}{}&
 \multicolumn{1}{c}{}&
 \multicolumn{1}{c}{}&
 \multicolumn{1}{c}{}&
 \multicolumn{1}{c}{}\\
 
 \cline{2-3}

 \multicolumn{1}{c|}{}&
 \textbf{H}&
 \multicolumn{1}{c}{}&
 \multicolumn{1}{c}{}&
 \multicolumn{1}{c}{}&
 \multicolumn{1}{c}{}&
 \multicolumn{1}{c}{}&
 \multicolumn{1}{c}{}&
 \multicolumn{1}{c}{}&
 \multicolumn{1}{c}{}&
 \multicolumn{1}{c}{}\\

\cline{2-2}
 
\end{longtable}

\bodyspacing

\paragraph{Step 3} Compare pairs of items. 

Each criterion is compared again
other alternatives to determine the relative important of the characteristic.

\paragraph{Step 4} Create the ranking of items. 

A ranking system is created
based on the relative occurrence of each element in the matrix.

The use of pairwise comparisons was particularly useful in the research context
as the method is ideal for ranking a set of decision-making criteria and rate
the criteria on a relative scale of importance.

\subsection{Summary}
\label{sec:exploratory-study:research-methods:summary}

This section has described the two primary research methods that were used
during the exploratory phase of this research. The combined effect of using the
two methods is appropriate as the exploratory study involved a series of
qualitative phases. The details of the study are outlined in Section~\ref{sec:exploratory-study:methodology}.