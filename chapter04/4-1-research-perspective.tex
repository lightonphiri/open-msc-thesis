\section{Research perspective}
\label{sec:exploratory-study:research-perspective}

\subsection{Prior research observations}
\label{sec:exploratory-study:research-perspective:prior-research-observations}

In our earlier work linked to this research, some issues that hinder
ubiquitous access to information and widespread preservation in Africa have
been highlighted \citep{Suleman2008}. A number of potential solutions to the
issues
raised have in the recent past also been presented, and take the form of
lightweight systems \citep{Suleman2007,Suleman2010a} with simplicity as
the key criterion.

However, the proposed solutions were solely based on specific user
requirements. The significance of prior work stems from the fact that they provided
this research with working hypotheses, which take the form of a set of
observable facts, that helped set the stage for the exploratory study.

\subsection{Research questions}
\label{sec:exploratory-study:research-perspective:research-questions}

The primary research question for this research, described in Section
\ref{sec:introduction:research-questions}, seeks to investigate the feasibility
of implementing \gls{dl} services that are based on simple
architectures. In order to better understand the simplicity of services, a
secondary research question, which was the main driving factor for the
exploratory study, was formulated as outlined below.

\begin{itemize}
 \item How should simplicity for \gls{dl} storage and service
architectures be defined?
\end{itemize}

The overall aim of the exploratory study was two-fold: firstly, it served to
guide the overall direction of the research, and secondly, it was aimed at
understanding contemporary \gls{dl} design in such a way as to be able to
better prescribe an alternative design approach that might result in simpler
\gls{dl} tools and services.

In order to obtain a reliable and comprehensive understanding of the desired
result, a qualitative study was conducted using a Grounded Theory approach.

\subsection{Summary}
\label{sec:exploratory-study:research-perspective:summary}

This section has highlighted prior work related to this research that
helped set the stage for the exploratory study. The section also outlined how
the exploratory study fits into the overall aims of the research by outlining
the rationale and significance of the study. In the subsequent section, the
research methods used during the exploratory study are discussed.