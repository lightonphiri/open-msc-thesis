% 
% Abstract iyambila pamene apa.

%
% check links below for nice way to start abstract
% http://research.cs.wisc.edu/htcondor/
% http://research.berkeley.edu/ucday/abstract.html
% 

% 
% Motivation/problem statement
% 
% It would seem as though glossary items are supposed to be expanded in main body of manuscript *** might have to look this up --some day
% 
The design of Digital Library Systems (DLSes) has evolved overtime, both in sophistication and complexity, to complement the complex nature and sheer size of digital content being curated. However, there is also a growing demand from content curators, with relatively small-size collections, for simpler and more manageable tools and services to manage their content. The reasons for this particular need are driven by the assumption that simplicity and manageability might ultimately translate to lower costs of maintenance of such systems.

This research proposes and advocates for a minimalist and simplistic approach to the overall design of DLSes. It is hypothesised that Digital Library (DL) tools and services based on such designs could potentially be easy to use and manage.

% 
% Methods/procedure/approach

% meta-analysis
% prototype simple repository
% real-world case studies
% developer survey
% performance experiments
% 
A meta-analysis of existing DL and non-DL tools was conducted to aid the derivation of design principles for simple DLSes. The design principles were then mapped to design decisions applied to the design of a prototype simple repository. In order to assess the effectiveness of the simple repository design, two real-world case study collections were implemented based on the design. In addition, a developer-oriented study was conducted using one of the case study collections to evaluate the simplicity and ease of use of the prototype system. Furthermore, performance experiments were conducted to establish the extent to which such a simple design approach would scale and also establish comparative advantages to existing designs.

% 
% Results/findings/product
% 
% case studies implementation
% developer survey
% performance benchmarks


% 
% Conclusions/implications
In general, the study outlined some possible implications of simplifying DLS design; specifically the results from the developer-oriented user study indicate that simplicity in the design of the DLS repository sub-layer does not severely impact the interaction between the service sub-layer and the repository sub-layer. Furthermore, the scalability experiments indicate that desirable performance results for small- and medium-sized collections are attainable.

The practical implication of the proposed design approach is two-fold: firstly the minimalistic design has the potential to be used to design simple and yet easy to use tools with comparable features to those exhibited by well-established DL tools; and secondly, the principled design approach has the potential to be applied to the design of non-DL application domains.