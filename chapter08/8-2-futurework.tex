\section{Future work}
\label{sec:conclusion:future-work}

%%%%%Future work idzakhala pompo\ldots

%%\subsection{Reference Implementation}
%%\label{sec:conclusion:future-work:refrence-implementation}

The objectives of this research were successfully achieved. However, there are still a number of potential research directions that could be further explored. The following are some potential future research areas that could be explored to complement the work conducted in this research. 

\subsection{Software packaging}
\label{sec:conclusion:future-work:software-packaging}

A key issue that has been linked to user adoption and overall usability of \gls{dl} software is the installation and configuration process associated to such systems \citep{Korber2008}. There have been a number of attempts to implement out-of-the-box systems \citep{Maly2004,Eprints3APT2011}. However, these have mostly been specific to particular operating system platforms. A potential research area could thus involve investigating how to simplify the packing of \gls{dl} tools and services.

\subsection{Version control}
\label{sec:conclusion:future-work:version-control}

The integration of digital object version control could significantly complement the preservation of resources stored in \glspl{dl}. This is an area that is already currently being explored \citep{DSpace3Versioning2012}. However, there is still a need to further explore how this important aspect of \gls{dl} preservation can be simplified.

\subsection{Reference implementation}
\label{sec:conclusion:future-work:reference-implementation}

The applicability of the design principles was presented in form of a simple prototype repository design. However, \glspl{dls} are multi-faceted applications and it would be interesting to design and implement a reference implementation composed of components---user interface and service layer components---designed using this prescribed design approach. This would further set the stage to conduct user studies aimed at determining whether simplifying the overall design of \glspl{dls} would have an impact on the way users interact with such systems. In addition, this would make it possible for desirable aspects of \glspl{dl}, for instance interoperability, to be evaluated as part of a complete system. Furthermore, a detailed evaluation of the integration of prominent \gls{dls}-specific standards and protocols with such a reference implementation would prove invaluable.