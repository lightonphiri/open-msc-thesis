\section[Repository design]{Repository design}
\label{sec:implementation:object-storage}

%%%%%Iyi subsection niya design rationale\ldots ma mapping ndi zochokela ku Exploratory Study.

\subsection[Design decisions]{Design decisions}
\label{sec:implementation:simple-repository:object-identifiers}

In Section~\ref{sec:background:software-design-decisions}, the significance of software design decisions were outlined; in addition prominent methods used to capture design decisions were highlighted. The design decisions associated with the architectural design of the repository sub-layer were arrived by taking into account the principles derived during the exploratory study (see Chapter~\ref{ch:exploratory-study}). Tables~\ref{tab:designing-for-simplicity:design-decisions:bitstream-storage},~\ref{tab:designing-for-simplicity:design-decisions:metadata-storage},~\ref{tab:designing-for-simplicity:design-decisions:object-naming-scheme} and~\ref{tab:designing-for-simplicity:design-decisions:object-structure} outline the detailed design decisions applied to design the repository.

\tablespacing
%%%%%\begin{longtable}{p{0.3\linewidth} p{0.6\linewidth}}
\begin{longtable}{
>{\arraybackslash}p{0.20\linewidth}|
>{\arraybackslash}p{0.70\linewidth}}

\caption{Simple repository persistent object store design decision}
\label{tab:designing-for-simplicity:design-decisions:bitstream-storage} \\

 %%%%%\toprule
 %%%%%\hline
 \textbf{Element} & \textbf{Description}\\
 %%%%%\midrule
 %%%%%\hline
 \cline{1-2}
 \endfirsthead

 \caption[]{(continued)}\\
 %%%%%\toprule
 %%%%%\hline
 \textbf{Element} & \textbf{Description}\\
 %%%%%\midrule
 %%%%%\hline
 \cline{1-2}
 \endhead

 % Page footer
 %%%%%\midrule
 %%%%%\hline
 \multicolumn{2}{r}{(Continued on next page)} \\
 \endfoot

 % Last page footer
 %%%%%\bottomrule
 \endlastfoot

 {\textbf{Issues}}&
 {Principles 1, 2, 6 and 8} \\

 %%%%%\cmidrule[0.1pt](l{0.5em}r{0.5em}){1-2}
 \cline{1-2}

 {\textbf{Decision}}&
 {Store bitstreams on the local operating system filesystem} \\

 %%%%%\cmidrule[0.1pt](l{0.5em}r{0.5em}){1-2}
 \cline{1-2}

 %%%%%{\textbf{Status}} &
 %%%%%{Approved} \\

 %%%%%\cmidrule[0.1pt](l{0.5em}r{0.5em}){1-2}
 %%%%%\cline{1-2}

 {\textbf{Assumptions}} &
 {None} \\

 %%%%%\cmidrule[0.1pt](l{0.5em}r{0.5em}){1-2}
 \cline{1-2}

 {\textbf{Alternatives}} &
 {Store bitstreams as blobs in a database; store bitstreams in the cloud} \\
 %%%%%\cmidrule[0.1pt](l{0.5em}r{0.5em}){1-2}
 \cline{1-2}

 {\textbf{Rationale}} &
 {Backup and migration tasks associate to repository objects can be potentially simplified; operating system commands can be used to perform repository management tasks} \\
 %%%%%\cmidrule[0.1pt](l{0.5em}r{0.5em}){1-2}
 \cline{1-2}

 {\textbf{Implications}} &
 {None --most conventional tools and services use the same approach} \\
 %%%%%\cmidrule[0.1pt](l{0.5em}r{0.5em}){1-2}
 \cline{1-2}

 {\textbf{Notes}} &
 {None} \\
 %%%%%\cmidrule[0.1pt](l{0.5em}r{0.5em}){1-2}
 %%%%%\cline{1-2}

 \end{longtable}

\bodyspacing

\tablespacing
%%%%%\begin{longtable}{p{0.3\linewidth} p{0.6\linewidth}}
\begin{longtable}{
>{\arraybackslash}p{0.20\linewidth}|
>{\arraybackslash}p{0.70\linewidth}}

\caption{Simple repository metadata storage design decision}
\label{tab:designing-for-simplicity:design-decisions:metadata-storage} \\

 %%%%%\toprule
 %%%%%\hline
 \textbf{Element} & \textbf{Description}\\
 %%%%%\midrule
 %%%%%\hline
 \cline{1-2}
 \endfirsthead

 \caption[]{(continued)}\\
 %%%%%\toprule
 %%%%%\hline
 \textbf{Element} & \textbf{Description}\\
 %%%%%\midrule
 %%%%%\hline
 \cline{1-2}
 \endhead

 % Page footer
 %%%%%\midrule
 %%%%%\hline
 \multicolumn{2}{r}{(Continued on next page)} \\
 \endfoot

 % Last page footer
 %%%%%\bottomrule
 \endlastfoot

 {\textbf{Issues}}&
 {Principles 1, 2, 5, 6 and 8} \\

 %%%%%\cmidrule[0.1pt](l{0.5em}r{0.5em}){1-2}
 \cline{1-2}

 {\textbf{Decision}}&
 {Native operating system filesystem used for metadata storage} \\

 %%%%%\cmidrule[0.1pt](l{0.5em}r{0.5em}){1-2}
 \cline{1-2}

 %%%%%{\textbf{Status}} &
 %%%%%{Approved} \\

 %%%%%\cmidrule[0.1pt](l{0.5em}r{0.5em}){1-2}
 %%%%%\cline{1-2}

 {\textbf{Assumptions}} &
 {None} \\

 %%%%%\cmidrule[0.1pt](l{0.5em}r{0.5em}){1-2}
 \cline{1-2}

 {\textbf{Alternatives}} &
 {Relational database; NoSQL database; embed metadata into digital objects} \\
 %%%%%\cmidrule[0.1pt](l{0.5em}r{0.5em}){1-2}
 \cline{1-2}

 {\textbf{Rationale}} &
 {Storing metadata records in plain text files ensures platform independence; complexities introduced by alternative third-party storage solution avoided through the use of native filesystem} \\
 %%%%%\cmidrule[0.1pt](l{0.5em}r{0.5em}){1-2}
 \cline{1-2}

 {\textbf{Implications}} &
 {No standard method for data access (e.g. SQL); Transaction process support only available via simple locking; non-availability of complex security mechanisms} \\
 %%%%%\cmidrule[0.1pt](l{0.5em}r{0.5em}){1-2}
 \cline{1-2}

 {\textbf{Notes}} &
 {None} \\
 %%%%%\cmidrule[0.1pt](l{0.5em}r{0.5em}){1-2}
 %%%%%\cline{1-2}

 \end{longtable}

\bodyspacing

\tablespacing
%%%%%\begin{longtable}{p{0.3\linewidth} p{0.6\linewidth}}
\begin{longtable}{
>{\arraybackslash}p{0.20\linewidth}|
>{\arraybackslash}p{0.70\linewidth}}

\caption{Simple repository object naming scheme design decision}
\label{tab:designing-for-simplicity:design-decisions:object-naming-scheme} \\

 %%%%%\toprule
 %%%%%\hline
 \textbf{Element} & \textbf{Description}\\
 %%%%%\midrule
 %%%%%\hline
 \cline{1-2}
 \endfirsthead

 \caption[]{(continued)}\\
 %%%%%\toprule
 %%%%%\hline
 \textbf{Element} & \textbf{Description}\\
 %%%%%\midrule
 %%%%%\hline
 \cline{1-2}
 \endhead

 % Page footer
 %%%%%\midrule
 %%%%%\hline
 \multicolumn{2}{r}{(Continued on next page)} \\
 \endfoot

 % Last page footer
 %%%%%\bottomrule
 \endlastfoot

 {\textbf{Issues}}&
 {Principle 5} \\

 %%%%%\cmidrule[0.1pt](l{0.5em}r{0.5em}){1-2}
 \cline{1-2}

 {\textbf{Decision}}&
 {Use actual object name as unique identifier} \\

 %%%%%\cmidrule[0.1pt](l{0.5em}r{0.5em}){1-2}
 \cline{1-2}

 %%%%%{\textbf{Status}} &
 %%%%%{Approved} \\

 %%%%%\cmidrule[0.1pt](l{0.5em}r{0.5em}){1-2}
 %%%%%\cline{1-2}

 {\textbf{Assumptions}} &
 {Native operating systems } \\

 %%%%%\cmidrule[0.1pt](l{0.5em}r{0.5em}){1-2}
 \cline{1-2}

 {\textbf{Alternatives}} &
 {File hash values; automatically generated identifiers} \\
 %%%%%\cmidrule[0.1pt](l{0.5em}r{0.5em}){1-2}
 \cline{1-2}

 {\textbf{Rationale}} &
 {Native operating systems ensure file naming uniqueness at directory level. In addition, it is a relatively simpler way of uniquely identifying objects as object naming control is given to end users, rather than imposing it on them} \\
 %%%%%\cmidrule[0.1pt](l{0.5em}r{0.5em}){1-2}
 \cline{1-2}

 {\textbf{Implications}} &
 {Object integrity has a potential to be compromised; objects could potentially be duplicated by simply renaming them} \\
 %%%%%\cmidrule[0.1pt](l{0.5em}r{0.5em}){1-2}
 \cline{1-2}

 {\textbf{Notes}} &
 {None} \\
 %%%%%\cmidrule[0.1pt](l{0.5em}r{0.5em}){1-2}
 %%%%%\cline{1-2}

 \end{longtable}

\bodyspacing

\tablespacing
%%%%%\begin{longtable}{p{0.3\linewidth} p{0.6\linewidth}}
\begin{longtable}{
>{\arraybackslash}p{0.20\linewidth}|
>{\arraybackslash}p{0.70\linewidth}}

\caption{Simple repository object storage structure design decision}
\label{tab:designing-for-simplicity:design-decisions:object-structure} \\

 %%%%%\toprule
 %%%%%\hline
 \textbf{Element} & \textbf{Description}\\
 %%%%%\midrule
 %%%%%\hline
 \cline{1-2}
 \endfirsthead

 \caption[]{(continued)}\\
 %%%%%\toprule
 %%%%%\hline
 \textbf{Element} & \textbf{Description}\\
 %%%%%\midrule
 %%%%%\hline
 \cline{1-2}
 \endhead

 % Page footer
 %%%%%\midrule
 %%%%%\hline
 \multicolumn{2}{r}{(Continued on next page)} \\
 \endfoot

 % Last page footer
 %%%%%\bottomrule
 \endlastfoot

 {\textbf{Issues}}&
 {Principles 6 and 7} \\

 %%%%%\cmidrule[0.1pt](l{0.5em}r{0.5em}){1-2}
 \cline{1-2}

 {\textbf{Decision}}&
 {Store bitstreams alongside metadata records --at the same directory level on the filesystem; filesystem directory to be used as container structures for repository objects} \\

 %%%%%\cmidrule[0.1pt](l{0.5em}r{0.5em}){1-2}
 \cline{1-2}

 %%%%%{\textbf{Status}} &
 %%%%%{Approved} \\

 %%%%%\cmidrule[0.1pt](l{0.5em}r{0.5em}){1-2}
 %%%%%\cline{1-2}

 {\textbf{Assumptions}} &
 {The other sub-layers of the \gls{dls} have read, write and execute access to the repository root node} \\

 %%%%%\cmidrule[0.1pt](l{0.5em}r{0.5em}){1-2}
 \cline{1-2}

 {\textbf{Alternatives}} &
 {Separate storage locations for bitstreams and metadata records} \\
 %%%%%\cmidrule[0.1pt](l{0.5em}r{0.5em}){1-2}
 \cline{1-2}

 {\textbf{Rationale}} &
 {Storing bitstreams and corresponding metadata records alongside each other could ultimately make potential migration processes easier; container structures could potentially make it easier to move repository objects across different platforms} \\
 %%%%%\cmidrule[0.1pt](l{0.5em}r{0.5em}){1-2}
 \cline{1-2}

 {\textbf{Implications}} &
 {None} \\
 %%%%%\cmidrule[0.1pt](l{0.5em}r{0.5em}){1-2}
 \cline{1-2}

 {\textbf{Notes}} &
 {None} \\
 %%%%%\cmidrule[0.1pt](l{0.5em}r{0.5em}){1-2}
 %%%%%\cline{1-2}

 \end{longtable}

\bodyspacing

\subsection[Architecture]{Architecture}
\label{sec:implementation:simple-repository:object-storage}

The architectural design is centred around designing a simple repository which at a bare minimum is capable of facilitating the core features of a \gls{dls}---long term preservation and ease of access to digital objects.

\tablespacing
%%%%%\begin{longtable}{p{0.3\linewidth} p{0.6\linewidth}}
\begin{longtable}{
>{\arraybackslash}p{0.25\linewidth}|
>{\arraybackslash}p{0.20\linewidth}|
>{\arraybackslash}p{0.45\linewidth}}

\caption{Simple repository component composition}
\label{tab:designing-for-simplicity:architecture:repository-components} \\

 %%%%%\toprule
 %%%%%\hline
 \textbf{Component} & 
 \textbf{File Type} & 
 \textbf{Description}\\
 %%%%%\midrule
 %%%%%\hline
 \cline{1-3}
 \endfirsthead

 \caption[]{(continued)}\\
 %%%%%\toprule
 %%%%%\hline
 \textbf{Component} & 
 \textbf{File Type} & 
 \textbf{Description}\\
 %%%%%\midrule
 %%%%%\hline
 \cline{1-3}
 \endhead

 % Page footer
 %%%%%\midrule
 %%%%%\hline
 \multicolumn{3}{r}{(Continued on next page)} \\
 \endfoot

 % Last page footer
 %%%%%\bottomrule
 \endlastfoot

 {\textbf{Container Object}}&
 {Directory} &
 {Structure used to store digital objects} \\

 %%%%%\cmidrule[0.1pt](l{0.5em}r{0.5em}){1-2}
 \cline{1-3}

 {\textbf{Content Object}} &
 {Regular file} &
 {Content/bitstreams to be stored in the repository} \\

 %%%%%\cmidrule[0.1pt](l{0.5em}r{0.5em}){1-2}
 \cline{1-3}

 {\textbf{Metadata Object}} &
 {Regular file} &
 {XML-encoded plain text file for storing metadata records} \\

 %%%%%\cmidrule[0.1pt](l{0.5em}r{0.5em}){1-2}
 %%%%%\cline{1-3}

 \end{longtable}

\bodyspacing

\begin{figure}[t]
 \centering
 \framebox[\textwidth]{
 % Generated with LaTeXDraw 2.0.8
% Tue Feb 26 11:01:16 SAST 2013
% \usepackage[usenames,dvipsnames]{pstricks}
% \usepackage{epsfig}
% \usepackage{pst-grad} % For gradients
% \usepackage{pst-plot} % For axes
\scalebox{1} % Change this value to rescale the drawing.
{
\begin{pspicture}(0,-6.66)(12.02,6.4)
\definecolor{color1109b}{rgb}{0.7176470588235294,0.8862745098039215,0.9411764705882353}
\psframe[linewidth=0.06,dimen=outer,fillstyle=solid,fillcolor=color1109b](11.6,2.38)(8.4,1.18)
\psline[linewidth=0.06cm](2.2,3.78)(2.2,-5.02)
\psframe[linewidth=0.06,dimen=outer,shadow=true,shadowangle=-45.0,shadowsize=0.1,fillstyle=solid,fillcolor=color1109b](4.4,2.58)(0.0,0.58)
\psframe[linewidth=0.06,dimen=outer,shadow=true,shadowangle=-45.0,shadowsize=0.1,fillstyle=solid,fillcolor=color1109b](4.4,-0.62)(0.0,-2.62)
\psframe[linewidth=0.06,dimen=outer,shadow=true,shadowangle=-45.0,shadowsize=0.1,fillstyle=solid,fillcolor=color1109b](4.4,5.77)(0.0,3.77)
\psframe[linewidth=0.06,dimen=outer,shadow=true,shadowangle=-45.0,shadowsize=0.1,fillstyle=solid,fillcolor=color1109b](4.4,-3.82)(0.0,-5.82)
\psframe[linewidth=0.06,dimen=outer,doubleline=true,doublesep=0.02,doublecolor=white,fillstyle=solid](3.8,5.38)(0.6,4.18)
\psframe[linewidth=0.06,dimen=outer](11.8,5.82)(8.2,1.0)
\usefont{T1}{ptm}{b}{n}
\rput(10.033906,5.445){KEY}
\psframe[linewidth=0.06,linestyle=dashed,dash=0.16cm 0.16cm,dimen=outer,fillstyle=solid](11.4,2.18)(8.6,1.38)
\usefont{T1}{ptm}{b}{n}
\rput(10.001562,1.785){Bitstream}
\usefont{T1}{ptm}{b}{n}
\rput(2.146875,4.785){/usr/local/x}
\psframe[linewidth=0.06,dimen=outer,fillstyle=solid](3.8,2.18)(0.6,0.98)
\usefont{T1}{ptm}{b}{n}
\rput(2.2579687,1.585){abcd}
\psframe[linewidth=0.06,dimen=outer,fillstyle=solid](3.8,-1.02)(0.6,-2.22)
\usefont{T1}{ptm}{b}{n}
\rput(2.218125,-1.615){efgh}
\psframe[linewidth=0.06,linestyle=dashed,dash=0.16cm 0.16cm,dimen=outer,fillstyle=solid](3.8,-4.22)(0.6,-5.42)
\usefont{T1}{ptm}{b}{n}
\rput(2.154375,-4.815){ijkl}
\psline[linewidth=0.06cm,tbarsize=0.07055555cm 35.0,bracketlength=0.1]{]-}(4.98,-1.62)(5.28,-1.62)
\usefont{T1}{ptm}{b}{n}
\rput(7.5448437,-1.315){/usr/local/x/abcd/efgh}
\usefont{T1}{ptm}{b}{n}
\rput(8.476875,-1.915){/usr/local/x/abcd/efgh.metadata}
\psline[linewidth=0.06cm,tbarsize=0.07055555cm 35.0,bracketlength=0.1]{]-}(5.0,-4.82)(5.32,-4.82)
\usefont{T1}{ptm}{b}{n}
\rput(7.857969,-4.515){/usr/local/x/abcd/efgh/ijkl}
\usefont{T1}{ptm}{b}{n}
\rput(8.786875,-5.115){/usr/local/x/abcd/efgh/ijkl.metadata}
\psframe[linewidth=0.06,dimen=outer,fillstyle=solid,fillcolor=color1109b](11.6,5.18)(8.4,3.98)
\psframe[linewidth=0.06,dimen=outer,doubleline=true,doublesep=0.02,doublecolor=white,fillstyle=solid](11.4,4.98)(8.6,4.18)
\usefont{T1}{ptm}{b}{n}
\rput(9.996718,4.585){Archive Root}
\psframe[linewidth=0.06,dimen=outer,fillstyle=solid,fillcolor=color1109b](11.6,3.78)(8.4,2.58)
\psframe[linewidth=0.06,dimen=outer,fillstyle=solid](11.4,3.58)(8.6,2.78)
\usefont{T1}{ptm}{b}{n}
\rput(9.98,3.185){Container}
\end{pspicture} 
}


 }
 \caption{Simple repository object structure}
 \label{fig:design:repository-design:repository-hierarchical-structure}
\end{figure}

The repository design is file-based and makes use of a typical native operating system filesystem as the core infrastructure. Table~\ref{tab:designing-for-simplicity:architecture:repository-components} shows the main components that make up the repository sub-layer, with all the components residing on the filesystem, arranged and organised as normal operating system files---regular files and/or directories---as shown in Figure~\ref{fig:design:repository-design:repository-hierarchical-structure}.

\begin{figure}[t]
 \centering
 \framebox[\textwidth]{
 % Generated with LaTeXDraw 2.0.8
% Sat Mar 02 11:36:57 SAST 2013
% \usepackage[usenames,dvipsnames]{pstricks}
% \usepackage{epsfig}
% \usepackage{pst-grad} % For gradients
% \usepackage{pst-plot} % For axes
\scalebox{1} % Change this value to rescale the drawing.
{
\begin{pspicture}(0,-5.84)(8.22,5.86)
\definecolor{color139b}{rgb}{0.9137254901960784,1.0,1.0}
\definecolor{color148b}{rgb}{0.7176470588235294,0.8862745098039215,0.9411764705882353}
\psframe[linewidth=0.04,linecolor=color139b,dimen=outer,shadow=true,shadowangle=-45.0,shadowsize=0.1,fillstyle=solid,fillcolor=color139b](7.6,5.2)(0.0,-5.2)
\psframe[linewidth=0.04,linecolor=color139b,dimen=outer,shadow=true,shadowangle=-45.0,shadowsize=0.1,fillstyle=solid,fillcolor=color139b](7.2,4.8)(0.4,-4.0)
\usefont{T1}{ptm}{b}{n}
\rput(2.3646874,-3.595){REPOSITORY}
\psframe[linewidth=0.04,linecolor=color139b,dimen=outer,shadow=true,shadowangle=-45.0,shadowsize=0.1,fillstyle=solid,fillcolor=color139b](6.8,4.4)(0.8,-3.0)
\usefont{T1}{ptm}{b}{n}
\rput(2.3885937,-2.595){COLLECTION}
\usefont{T1}{ptm}{b}{n}
\rput(2.3026562,-4.695){FILESYSTEM}
\psframe[linewidth=0.06,dimen=outer,fillstyle=solid,fillcolor=color148b](6.4,4.0)(4.0,-1.8)
\psframe[linewidth=0.06,dimen=outer,fillstyle=solid,fillcolor=color148b](3.6,4.0)(1.2,-1.8)
\psframe[linewidth=0.06,dimen=outer,fillstyle=hlines*,hatchwidth=0.06,hatchangle=0.0,hatchsep=0.1](6.2,3.7)(4.2,3.1)
\psframe[linewidth=0.06,dimen=outer,fillstyle=hlines*,hatchwidth=0.06,hatchangle=0.0,hatchsep=0.1](6.2,2.9)(4.2,2.3)
\psframe[linewidth=0.06,dimen=outer,fillstyle=hlines*,hatchwidth=0.06,hatchangle=0.0,hatchsep=0.1](6.2,2.1)(4.2,1.5)
\psframe[linewidth=0.06,dimen=outer,fillstyle=hlines*,hatchwidth=0.06,hatchangle=0.0,hatchsep=0.1](6.2,1.3)(4.2,0.7)
\psframe[linewidth=0.06,dimen=outer,fillstyle=hlines*,hatchwidth=0.06,hatchangle=0.0,hatchsep=0.1](6.2,0.5)(4.2,-0.1)
\psframe[linewidth=0.06,dimen=outer,fillstyle=hlines*,hatchwidth=0.06,hatchangle=0.0,hatchsep=0.1](6.2,-0.3)(4.2,-0.9)
\usefont{T1}{ptm}{b}{n}
\rput(5.1734376,-1.395){Metadata}
\psframe[linewidth=0.06,dimen=outer,fillstyle=crosshatch*,hatchwidth=0.06,hatchangle=0.0,hatchsep=0.1](3.4,3.7)(1.4,3.1)
\psframe[linewidth=0.06,dimen=outer,fillstyle=crosshatch*,hatchwidth=0.06,hatchangle=0.0,hatchsep=0.1](3.4,2.9)(1.4,2.3)
\psframe[linewidth=0.06,dimen=outer,fillstyle=crosshatch*,hatchwidth=0.06,hatchangle=0.0,hatchsep=0.1](3.4,2.1)(1.4,1.5)
\psframe[linewidth=0.06,dimen=outer,fillstyle=crosshatch*,hatchwidth=0.06,hatchangle=0.0,hatchsep=0.1](3.4,1.3)(1.4,0.7)
\psframe[linewidth=0.06,dimen=outer,fillstyle=crosshatch*,hatchwidth=0.06,hatchangle=0.0,hatchsep=0.1](3.4,0.5)(1.4,-0.1)
\psframe[linewidth=0.06,dimen=outer,fillstyle=crosshatch*,hatchwidth=0.06,hatchangle=0.0,hatchsep=0.1](3.4,-0.3)(1.4,-0.9)
\usefont{T1}{ptm}{b}{n}
\rput(2.3765626,-1.395){Objects}
\psline[linewidth=0.06cm,arrowsize=0.113cm 2.5,arrowlength=1.4,arrowinset=0.4]{->}(3.44,3.4)(4.2,3.4)
\psline[linewidth=0.06cm,arrowsize=0.113cm 2.5,arrowlength=1.4,arrowinset=0.4]{->}(3.44,2.58)(4.2,2.58)
\psline[linewidth=0.06cm,arrowsize=0.113cm 2.5,arrowlength=1.4,arrowinset=0.4]{->}(3.44,1.76)(4.2,1.76)
\psline[linewidth=0.06cm,arrowsize=0.113cm 2.5,arrowlength=1.4,arrowinset=0.4]{->}(3.44,0.98)(4.2,0.98)
\psline[linewidth=0.06cm,arrowsize=0.113cm 2.5,arrowlength=1.4,arrowinset=0.4]{->}(3.44,0.2)(4.2,0.2)
\psline[linewidth=0.06cm,arrowsize=0.113cm 2.5,arrowlength=1.4,arrowinset=0.4]{->}(3.44,-0.62)(4.2,-0.62)
\end{pspicture} 
}


 }
 \caption{Simple repository object structure}
 \label{fig:design:repository-design:repository-object-structure}
\end{figure}

As shown in Figure~\ref{fig:design:repository-design:repository-hierarchical-structure}, a typical \gls{dls} repository would be located in an application accessible base root directory node, and is composed of two types of digital objects---Container Objects and Content Objects---both of which are created and stored within the repository with companion Metadata Objects that store representational information associated with the object. Figure~\ref{fig:design:repository-design:repository-object-structure} illustrates how Container and Content objects are stored on a typical filesystem.

Container Objects can be recursively created within the root node as the repository scales, and exhibit an interesting characteristic of a enabling the creation of additional Container Objects within them. As shown in Figure~\ref{fig:design:repository-design:container-object}, the Metadata Object associated with Container Objects holds information that uniquely identifies the object; optionally describe the object in more detail, including relationships that might exist with other objects within the repository; and a detailed log of objects contained within it---the manifest.

Content Objects represent digital objects---typically bitstreams---to be stored within the repository. As shown in Figure~\ref{fig:design:repository-design:digital-object}, the representational information stored in the Metadata Objects associated with Content Objects is similar to that of Container Objects, with the exception of manifest related information.

\begin{figure}
 \centering
 \framebox[\textwidth]{
 % Generated with LaTeXDraw 2.0.8
% Sat Mar 02 11:56:02 SAST 2013
% \usepackage[usenames,dvipsnames]{pstricks}
% \usepackage{epsfig}
% \usepackage{pst-grad} % For gradients
% \usepackage{pst-plot} % For axes
\scalebox{1} % Change this value to rescale the drawing.
{
\begin{pspicture}(0,-5.18)(12.64,5.14)
\definecolor{color119b}{rgb}{0.9137254901960784,1.0,1.0}
\definecolor{color123b}{rgb}{0.7176470588235294,0.8862745098039215,0.9411764705882353}
\psframe[linewidth=0.04,linecolor=color119b,dimen=outer,shadow=true,shadowangle=-45.0,shadowsize=0.1,fillstyle=solid,fillcolor=color119b](6.8,4.9)(0.0,-4.9)
\usefont{T1}{ptm}{b}{n}
\rput(3.4264061,4.505){CONTAINER OBJECT}
\psframe[linewidth=0.06,dimen=outer,shadow=true,shadowangle=-45.0,fillstyle=solid,fillcolor=color123b](5.6,4.1)(1.2,-2.5)
\psframe[linewidth=0.06,dimen=outer,fillstyle=solid](5.0,3.3)(1.8,2.1)
\psframe[linewidth=0.06,linestyle=dashed,dash=0.16cm 0.16cm,dimen=outer,fillstyle=solid](5.0,1.9)(1.8,0.7)
\psframe[linewidth=0.06,linestyle=dashed,dash=0.16cm 0.16cm,dimen=outer,fillstyle=solid](5.0,0.5)(1.8,-0.7)
\psframe[linewidth=0.06,dimen=outer,fillstyle=solid](5.0,-0.9)(1.8,-2.1)
\usefont{T1}{ptm}{b}{n}
\rput(3.414375,3.705){METADATA}
\psframe[linewidth=0.06,dimen=outer,shadow=true,shadowangle=-45.0,fillstyle=solid,fillcolor=color123b](5.6,-2.8)(1.2,-4.6)
\psframe[linewidth=0.06,dimen=outer,fillstyle=vlines*,hatchwidth=0.06,hatchangle=0.0,hatchsep=0.16](5.0,-3.64)(1.8,-4.36)
\usefont{T1}{ptm}{b}{n}
\rput(3.374375,-3.195){CONTAINER}
\usefont{T1}{ptm}{b}{n}
\rput(3.4235938,2.705){Identifier}
\usefont{T1}{ptm}{b}{n}
\rput(3.4532812,1.505){Descriptive}
\usefont{T1}{ptm}{b}{n}
\rput(3.4734375,1.105){Metadata}
\usefont{T1}{ptm}{b}{n}
\rput(3.5090625,0.105){Relationship}
\usefont{T1}{ptm}{b}{n}
\rput(3.46125,-0.295){Definitions}
\usefont{T1}{ptm}{b}{n}
\rput(3.3745313,-1.495){Manifest}
\psline[linewidth=0.06cm,tbarsize=0.07055555cm 100.0,bracketlength=0.07]{]-}(8.0,0.9)(8.6,0.9)
\usefont{T1}{ptm}{b}{n}
\rput(10.524844,1.105){XML Encoded}
\usefont{T1}{ptm}{b}{n}
\rput(10.768281,-3.395){Operating System}
\usefont{T1}{ptm}{b}{n}
\rput(10.684844,-3.895){Directory/Folder}
\usefont{T1}{ptm}{b}{n}
\rput(10.511094,0.605){Plain Text File}
\psline[linewidth=0.06cm,tbarsize=0.07055555cm 30.0,bracketlength=0.21]{]-}(8.0,-3.7)(8.6,-3.7)
\end{pspicture} 
}


 }
 \caption{Simple repository container object component structure}
 \label{fig:design:repository-design:container-object}
\end{figure}

\begin{figure}
 \centering
 \framebox[\textwidth]{
 % Generated with LaTeXDraw 2.0.8
% Sat Mar 02 11:56:10 SAST 2013
% \usepackage[usenames,dvipsnames]{pstricks}
% \usepackage{epsfig}
% \usepackage{pst-grad} % For gradients
% \usepackage{pst-plot} % For axes
\scalebox{1} % Change this value to rescale the drawing.
{
\begin{pspicture}(0,-4.48)(12.64,4.44)
\definecolor{color93b}{rgb}{0.9137254901960784,1.0,1.0}
\definecolor{color97b}{rgb}{0.7176470588235294,0.8862745098039215,0.9411764705882353}
\psframe[linewidth=0.04,linecolor=color93b,dimen=outer,shadow=true,shadowangle=-45.0,shadowsize=0.1,fillstyle=solid,fillcolor=color93b](6.8,4.2)(0.0,-4.2)
\usefont{T1}{ptm}{b}{n}
\rput(3.5064063,3.805){CONTENT OBJECT}
\psframe[linewidth=0.06,dimen=outer,shadow=true,shadowangle=-45.0,shadowsize=0.1,fillstyle=solid,fillcolor=color97b](5.6,3.4)(1.2,-1.8)
\psframe[linewidth=0.06,dimen=outer,fillstyle=solid](5.0,2.6)(1.8,1.4)
\psframe[linewidth=0.06,linestyle=dashed,dash=0.16cm 0.16cm,dimen=outer,fillstyle=solid](5.0,1.2)(1.8,0.0)
\psframe[linewidth=0.06,linestyle=dashed,dash=0.16cm 0.16cm,dimen=outer,fillstyle=solid](5.0,-0.2)(1.8,-1.4)
\usefont{T1}{ptm}{b}{n}
\rput(3.414375,3.005){METADATA}
\psframe[linewidth=0.06,dimen=outer,shadow=true,shadowangle=-45.0,shadowsize=0.1,fillstyle=solid,fillcolor=color97b](5.6,-2.1)(1.2,-3.9)
\psframe[linewidth=0.06,dimen=outer,fillstyle=crosshatch*,hatchwidth=0.06,hatchangle=0.0,hatchsep=0.16](5.0,-2.94)(1.8,-3.66)
\usefont{T1}{ptm}{b}{n}
\rput(3.3829687,-2.595){BITSTREAM}
\usefont{T1}{ptm}{b}{n}
\rput(3.4235938,2.005){Identifier}
\usefont{T1}{ptm}{b}{n}
\rput(3.4532812,0.805){Descriptive}
\usefont{T1}{ptm}{b}{n}
\rput(3.4734375,0.405){Metadata}
\usefont{T1}{ptm}{b}{n}
\rput(3.5090625,-0.595){Relationship}
\usefont{T1}{ptm}{b}{n}
\rput(3.46125,-0.995){Definitions}
\psline[linewidth=0.06cm,tbarsize=0.07055555cm 80.0,bracketlength=0.09]{]-}(8.0,0.8)(8.6,0.8)
\usefont{T1}{ptm}{b}{n}
\rput(10.524844,1.105){XML Encoded}
\usefont{T1}{ptm}{b}{n}
\rput(10.141562,-2.795){Bitstream}
\usefont{T1}{ptm}{b}{n}
\rput(10.585625,-3.295){(e.g. jpeg image)}
\usefont{T1}{ptm}{b}{n}
\rput(10.511094,0.605){Plain Text File}
\psline[linewidth=0.06cm,tbarsize=0.07055555cm 30.0,bracketlength=0.21]{]-}(8.0,-3.0)(8.6,-3.0)
\end{pspicture} 
}


 }
 \caption{Simple repository digital object component structure}
 \label{fig:design:repository-design:digital-object}
\end{figure}

\subsection{Summary}
\label{sec:implementation:simple-repository:summary}

In this chapter, the design of a prototype simple repository sub-layer was outlined through the mapping of design decisions and principles derived in Chapter~\ref{ch:exploratory-study}.